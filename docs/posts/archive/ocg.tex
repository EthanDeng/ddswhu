\documentclass{article}

\usepackage[colorlinks,linkcolor=blue]{hyperref}% do NOT set [ocgcolorlinks] here!
\usepackage[ocgcolorlinks,tikz]{ocgx2}
\usepackage{times}

\title{Exercises for Data Analysis (Python)}
\author{\href{https://ddswhu.me/}{ddswhu}}
\linespread{1.3}
\date{\today}



\begin{document}
\maketitle


\section{101 NumPy Exercises for Data Analysis (Python)}


\indent The goal of the numpy exercises is to serve as a reference as well as to get you to apply numpy beyond the basics. The questions are of 4 levels of difficulties with L1 being the easiest to L4 being the hardest.

\noindent\textbf{Question 1}: How to access the response body of requests?\\
\switchocg{ocg1}{\textbf{Show answer}}

\begin{ocg}{Python Code}{ocg1}{1}
\begin{verbatim}
import requests
url = 'www.example.com'
reponse = request.get(url=url).content
print(response)
\end{verbatim}
\end{ocg}

\section{Using ocg in TikZ}

The goal of the numpy exercises is to serve as a reference as well as to get you to apply numpy beyond the basics \switchocg{ocgridid}{Grid}. The questions are of 4 levels of difficulties with L1 being the easiest to L4 being the hardest  \switchocg{ocstatesid}{States}. If we don't understand \switchocg{ocedgesid}{Edge}.
\begin{figure}[!h]
\centering
\begin{tikzpicture}[node distance=3cm, state/.style={fill=green!20},auto]

\begin{ocg}{grid}{ocgridid}{1}
\draw[black!20] (-1,-1) grid (4,2);
\end{ocg}

\begin{ocg}{states}{ocstatesid}{1}
\node[state] (q_a) {$q_a$};
\node[state] (q_b) [right of=q_a] {$q_b$};
\end{ocg}

\begin{ocg}{edges}{ocedgesid}{1}
\path[->]
(q_a) edge node {0} (q_b)
edge [loop above] node {0} ()
(q_b) edge [loop above] node {1} ();
\end{ocg}

\end{tikzpicture}
\end{figure}

\end{document}
